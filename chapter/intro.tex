\subsection{Introduction to NcX channel}
The sodium-calcium exchanger (often denoted Na+/Ca2+ exchanger, exchange protein, or NCX) is an antiporter membrane protein that removes calcium from cells. It uses the energy that is stored in the electrochemical gradient of sodium (Na+) by allowing Na+ to flow down its gradient across the plasma membrane in exchange for the countertransport of calcium ions (Ca2+). A single calcium ion is exported for the import of three sodium ions. The exchanger exists in many different cell types and animal species. The NCX is considered one of the most important cellular mechanisms for removing Ca2+.

The exchanger is usually found in the plasma membranes and the mitochondria and endoplasmic reticulum of excitable cells. [wikipedia]

\subsection{Introduction to Markov Model}
In probability theory, a Markov model is a stochastic model used to model randomly changing systems. It is assumed that future states depend only on the current state, not on the events that occurred before it (that is, it assumes the Markov property). Generally, this assumption enables reasoning and computation with the model that would otherwise be intractable. For this reason, in the fields of predictive modelling and probabilistic forecasting, it is desirable for a given model to exhibit the Markov property. [wikipwdia]


\newpage

\subsection{This is a equation}
\bea
\alpha |0> + \beta |k>
\eea

\subsection{This is a matrix}

\bea
\begin{bmatrix}
a & b\\
c & d\\
\end{bmatrix}
\eea


